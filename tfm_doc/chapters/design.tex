\chapter{Diseño e implementación}
\label{cha:regulation}

\section{Aplicación web de 3 capas}

La aplicación \kode{sample\_app\_rails\_4}, escrita en lenguaje Ruby\footnotettt{Ruby}{https://www.ruby-lang.org/es}, es parte de un tutorial sobre el uso del \textit{framework} de desarrollo de aplicaciones web Ruby on Rails\footnotettt{Ruby on Rails}{http://www.rubyonrails.org.es}. Se trata de una aplicación de muestra, desarrollada mediante una combinación de simulaciones, pruebas de desarrollo \textit{TDD} y pruebas de integración. Tiene una arquitectura de 3 capas: cliente, aplicación y base de datos. Está creada a partir de páginas estáticas con contenido dinámico, tiene un diseño de sitio web, un modelo de datos de usuario y un sistema completo de registro y autenticación, incluida la activación de cuentas y restablecimientos de contraseñas. Además tiene funciones de \textit{microblogging} y redes sociales. Así, tendrá usuarios que crearán \textit{microposts} dentro de un marco de autenticación e inicio de sesión completo.

\begin{figure}[H]
\image{images/figures/sampleapp.png}
\caption{Integración entre características y componentes de Ruby on Rails.\label{fig:figure_placement_example}}
\end{figure}

La arquitectura de Ruby on Rails que implementa la aplicación web tiene las siguientes características:
\begin{itemize}
\item Arquitectura Modelo-Vista-Controlador (MVC): Mejora la capacidad de mantenimiento, desacoplamiento y pruebas de la aplicación.
\subitem Modelo: Lógica de negocio de la aplicación y reglas para manipular los datos. Representa la información en la base de datos y realizan las validaciones apropiadas.
\subitem Vista: Representa la interfaz de usuario.
\subitem Controlador: Responde a eventos e invoca perticiones al modelo cuando se hace una solicitud de información.
\item Arquitectura \textit{Representational State Transfer} (REST) para servicios web.
\item Soporta las principales bases de datos como MySQL\footnotettt{MySQL}{https://www.mysql.com}, Oracle\footnotettt{Oracle}{https://www.oracle.com/es/database/index.html} y PostgreSQL\footnotettt{PostgreSQL}{http://www.postgresql.org.es}, entre otras.
\item Lenguaje \textit{scripting} del lado del servidor de código abierto.
\item Convención sobre configuración.
\item Generadores de \textit{scripts} para automatizar tareas.
\item Uso de la máquina \textit{YAML}, formato de serialización de datos legible por humanos.
\end{itemize}

Las características anteriormente descritas se distribuyen en los siguientes componentes de Rails:
\begin{itemize}
\item \textit{Action Mailer}: Responsable de proporcionar servicios de correo electrónico. 
\item \textit{Action Pack}: Capta las solicitudes de usuario realizadas por el navegador y asignan estas solicitudes a acciones definidas en la capa de controladores.
\subitem \textit{Action Controller}: Enruta solicitudes a su controlador correspondiente. 
\subitem \textit{Action Dispatcher}: Controla el enrutamiento de la solicitud del navegador web, la analiza y procesa.
\subitem \textit{Action View}: Realiza la presentación de la página web solicitada.
\item \textit{Active Model}: Define la interfaz entre el \textit{Action Pack} y los módulos \textit{Active Record}.
\item \textit{Active Record}: Proporciona capacidad de crear relaciones o asociaciones entre modelos y construye la capa Modelo que conecta las tablas de la base de datos con su representación en las clases Ruby.
\item \textit{Active Resource}: Administra la conexión entre servicios web \textit{RESTful} y objetos de negocio.
\item \textit{Active Support}: Colección de clases de utilidad y extensiones de bibliotecas estándar de Ruby útiles para el desarrollo en Ruby on Rails.
\item \textit{Railties}: Código básico de Rails que construye nuevas aplicaciones. \end{itemize}	 

\begin{figure}[H]
\image{images/figures/rubyonrails.png}
\caption{Integración entre características y componentes de Ruby on Rails.\label{fig:figure_placement_example}}
\end{figure}

\section{Despliegue monomáquina}

La aplicación \kode{sample\_app\_rails\_4} está disponible en un repositorio GitHub\footnotettt{GitHub}{https://github.com}. Para poder trabajar con ella se hace una copia del repositorio a través de un \textit{fork}. Esto crea una bifurcación que permite la libre experimentación de cambios sin afectar el proyecto original y utilizar el proyecto de otra persona como punto de partida de una idea propia. Para hacer el \textit{fork} hay que seleccionar el botón con el mismo nombre y automáticamente se crea la copia del repositorio en la cuenta personal.

Ahora se crea una clonación local del fork para trabajar localmente:

\begin{codelisting}
\label{code:forkclone}
\codecaption{Clone del proyecto en local}
\begin{code}
\$ git clone https://github.com/CarolinaSantana/sample_app_rails_4-1.git
\end{code}
\end{codelisting}

Luego se entra en la carpeta local que lo contiene, seleccionando como \textit{gemset}, conjunto aislado de gemas incorporadas en la aplicación para la versión Ruby en uso por ella, 2.0.0, quedando como \kode{2.0.0@railstutorial\_rails\_4\_0}. Además se deja lista la base de datos de ejemplo que trae, con la intención de probar que los test que comprueban su funcionalidad pasen positivamente: 

\begin{codelisting}
\label{code:forkclone}
\codecaption{Clone del proyecto en local y comprobación de tests.}
\begin{code}
$ cd sample_app_rails_4-1
$ cp config/database.yml.example config/database.yml
$ bundle install --without production
$ bundle exec rake db:migrate
$ bundle exec rake db:test:prepare
$ bundle exec rspec spec/
\end{code}
\end{codelisting}

