\chapter{Diseño e implementación}
\label{cha:regulation}

\section{Aplicación web de 3 capas}

La aplicación \kode{sample\_app\_rails\_4}, escrita en lenguaje Ruby\footnotettt{Ruby}{https://www.ruby-lang.org/es}, es parte de un tutorial sobre el uso del \textit{framework} de desarrollo de aplicaciones web Ruby on Rails\footnotettt{Ruby on Rails}{http://www.rubyonrails.org.es}. Se trata de una aplicación de muestra, desarrollada mediante una combinación de simulaciones, pruebas de desarrollo \textit{TDD} y pruebas de integración. Tiene una arquitectura de 3 capas: cliente, aplicación y base de datos SQLite\footnotettt{SQLite}{https://sqlite.org}. Está creada a partir de páginas estáticas con contenido dinámico, tiene un diseño de sitio web, un modelo de datos de usuario y un sistema completo de registro y autenticación, incluida la activación de cuentas y restablecimientos de contraseñas. Además tiene funciones de \textit{microblogging} y redes sociales. Así, tendrá usuarios que crearán \textit{microposts} dentro de un marco de autenticación e inicio de sesión completo.

\begin{figure}[H]
\image{images/figures/sampleapp.png}
\caption{Integración entre características y componentes de Ruby on Rails.\label{fig:figure_placement_example}}
\end{figure}

La arquitectura de Ruby on Rails que implementa la aplicación web tiene las siguientes características:
\begin{itemize}
\item Arquitectura Modelo-Vista-Controlador (MVC): Mejora la capacidad de mantenimiento, desacoplamiento y pruebas de la aplicación.
\subitem-- Modelo: Lógica de negocio de la aplicación y reglas para manipular los datos. Representa la información en la base de datos y realizan las validaciones apropiadas.
\subitem-- Vista: Representa la interfaz de usuario.
\subitem-- Controlador: Responde a eventos e invoca perticiones al modelo cuando se hace una solicitud de información.
\item Arquitectura \textit{Representational State Transfer} (REST) para servicios web.
\item Soporta las principales bases de datos como MySQL\footnotettt{MySQL}{https://www.mysql.com}, Oracle\footnotettt{Oracle}{https://www.oracle.com/es/database/index.html} y PostgreSQL\footnotettt{PostgreSQL}{http://www.postgresql.org.es}, entre otras.
\item Lenguaje \textit{scripting} del lado del servidor de código abierto.
\item Convención sobre configuración.
\item Generadores de \textit{scripts} para automatizar tareas.
\item Uso de \textit{YAML}, formato de serialización de datos inspirado en lenguajes como \textit{XML} y \textit{C}.
\end{itemize}

Las características anteriormente descritas se distribuyen en los siguientes componentes de Rails:
\begin{itemize}
\item \textit{Action Mailer}: Responsable de proporcionar servicios de correo electrónico. 
\item \textit{Action Pack}: Capta las solicitudes de usuario realizadas por el navegador y asignan estas solicitudes a acciones definidas en la capa de controladores.
\subitem-- \textit{Action Controller}: Enruta solicitudes a su controlador correspondiente. 
\subitem-- \textit{Action Dispatcher}: Controla el enrutamiento de la solicitud del navegador web, la analiza y procesa.
\subitem-- \textit{Action View}: Realiza la presentación de la página web solicitada.
\item \textit{Active Model}: Define la interfaz entre el \textit{Action Pack} y los módulos \textit{Active Record}.
\item \textit{Active Record}: Proporciona capacidad de crear relaciones o asociaciones entre modelos y construye la capa Modelo que conecta las tablas de la base de datos con su representación en las clases Ruby.
\item \textit{Active Resource}: Administra la conexión entre servicios web \textit{RESTful} y objetos de negocio.
\item \textit{Active Support}: Colección de clases de utilidad y extensiones de bibliotecas estándar de Ruby útiles para el desarrollo en Ruby on Rails.
\item \textit{Railties}: Código básico de Rails que construye nuevas aplicaciones. \end{itemize}	 

\begin{figure}[H]
\image{images/figures/rubyonrails.png}
\caption{Integración entre características y componentes de Ruby on Rails.\label{fig:figure_placement_example}}
\end{figure}

\section{Despliegue monomáquina con contenedores}

La aplicación \kode{sample\_app\_rails\_4} está disponible en un repositorio GitHub\footnotettt{GitHub}{https://github.com}. Para trabajar con ella se hace una copia del repositorio a través de un \textit{fork}. Esto crea una bifurcación que permite la libre experimentación de cambios sin afectar el proyecto original y utilizar el proyecto de otra persona como punto de partida de una idea propia. Para hacer el \textit{fork} hay que seleccionar el botón con el mismo nombre y automáticamente se crea la copia del repositorio en la cuenta personal. Seguidamente, se clona esta copia para trabajar localmente:

%= lang:bash
\begin{code}
$ git clone https://github.com/CarolinaSantana/sample_app_rails_4-1.git 
\end{code}

Luego se entra en la carpeta local que lo contiene, seleccionando como \textit{gemset}, conjunto aislado de gemas incorporadas en la aplicación para la versión Ruby en uso, 2.0.0, quedando como \kode{2.0.0@railstutorial\_rails\_4\_0}. Además se deja lista la configuración de base de datos, utilizando la que viene a modo de ejemplo, con la intención de probar que los test que comprueban su funcionalidad pasen positivamente: 

%= lang:bash
\begin{code}
$ cd sample_app_rails_4-1
$ cp config/database.yml.example config/database.yml
$ bundle install --without production
$ bundle exec rake db:migrate
$ bundle exec rake db:test:prepare
$ bundle exec rspec spec/
\end{code}

\subsection{Docker}

Una vez completada la fase de preparación de la aplicación en el entorno local se quiere desplegar la siguiente estructura:
\begin{figure}[H]
\image{images/figures/monomachine-docker.png}
\caption{Estructura del despliegue monomáquina con contenedores Docker.\label{fig:figure_monomachine_docker}}
\end{figure}

De este modo, se obtienen las imágenes pertenecientes, mediante Docker, a un servicio que implementa la funcionalidad de la base de datos \kode{PostgreSQL} y un servicio que implementa el servidor web y  proxy inverso \kode{Nginx}. El proxy inverso es un proxy que aparenta ser un servidor web ante los clientes, pero que en realidad reenvía las solicitudes que recibe a uno o más servidores de origen.

Con la intención de implementar la base de datos de la aplicación desde un contenedor Docker se elimina del fichero \kode{Gemfile} la gema \kode{sqlite3 1.3.8}, como base de datos en desarrollo, y se cambia por una base de datos PosgreSQL, concretamente la versión \kode{pg 0.15.1}. Para instalar la nueva gema se ejecuta:

%= lang:bash
\begin{code}
$ bundle install --without production
\end{code}

La instalación de \kode{PostgreSQL} requiere credenciales de usuario con contraseña para acceder. La manera de especificarlo es mediante las variables de entorno \kode{\$POSTGRES\_USER} y \kode{\$POSTGRES\_PASSWORD} en el fichero local \kode{\textasciitilde{}/.postgres/credentials} cuyo contenido es:

\begin{codelisting}
\label{code:credentials}
\codecaption{Fichero \kode{\textasciitilde{}/.postgres/credentials}}
\begin{code}
export POSTGRES_USER=postgres
export POSTGRES_PASSWORD=postgres
\end{code}
\end{codelisting}

Se da permiso de ejecución tanto al fichero como a la carpeta que lo contiene y se exportan las variables de entorno en el directorio de la aplicación:

%= lang:bash
\begin{code}
$ chmod 0700 ~/.postgres
$ chmod 0700 ~/.postgres/credentials
$ . ~/.postgres/credentials
\end{code}

Ahora hay que cambiar el fichero de configuración de la base de datos de forma que se especifique lo siguiente:

\begin{codelisting}
\label{code:database}
\codecaption{Cambios en el fichero \kode{config/database.yml}}
\begin{code}
default: &default
  adapter: postgresql
  encoding: unicode
  pool: 5
  username: <%= ENV['POSTGRES_USER'] %>
  password: <%= ENV['POSTGRES_PASSWORD'] %>
  port: 5432

development:
  <<: *default
  database: sample_app_development  
  host : db

test:
  <<: *default
  database: sample_app_test
  host: db

production:
  <<: *default
  database: sample_app_production
\end{code}
\end{codelisting}

De esta manera se ha indicado a las bases de datos de desarrollo, test y producción el nombre de usuario y contraseña, que capturará a partir de las variables de entorno, y que ha de conectarse a \textit{db}, que será el nombre por el que ha de descubrir el contenedor docker que provisiona el servidor \kode{PostgreSQL}.

Por motivos de seguridad el fichero de configuración de la base de datos no se ha de subir al repositorio remoto, por lo que se hace una copia para tener esta configuración de ejemplo:

%= lang:bash
\begin{code}
$ cp config/database.yml config/database.yml.postgres
\end{code}

Por su parte, para poder acceder a la aplicación en cuestión, se cambia el servidor web \kode{rails s} por \kode{puma}\footnotettt{Servidor Web Puma}{http://puma.io}, construido para ofrecer mayor velocidad y paralelismo que, por ejemplo, el primero.

Para ello se añade al fichero \kode{Gemfile} la siguiente entrada:

\begin{codelisting}
\label{code:addpuma}
\codecaption{Fichero \kode{Gemfile}}
%= lang:ruby
\begin{code}
.
gem 'puma'
.
\end{code}
\end{codelisting}

Luego se instala la nueva gema: 

%= lang:bash
\begin{code}
$ bundle install
\end{code}

En este momento ya se tienen listas las configuraciones necesarias para proceder a la creación de los contenedores.

Como se ha comentado, a través de Docker se pueden obtener las imágenes de los servidores \kode{PostgreSQL} y \kode{Nginx}, pero para poder crear un contenedor que contenga la aplicación web hay que crear su imagen Docker. Para ello es necesario crear un fichero \kode{Dockerfile} que especifica cómo crearla, indicando que lo hará a partir de una imagen Ruby y que debe actualizar todos los paquetes e instalar \textit{nodejs}, necesario para la aplicación. Finalmente se le indica que haga una limpieza de las capas intermedias. Esta imagen se llamará \kode{sample\_app\_image\_0\_1}. 

\begin{codelisting}
\label{code:dockerfile}
\codecaption{Contenido de \kode{Dockerfile}}
%= lang:ruby
\begin{code}
##################################################
# Dockerfile to build a sample_app_rails_4 image #
##################################################
FROM ruby:2.0-onbuild
LABEL sample_app_image_0_1.version="0.1" sample_app_image_0_1.release-date="2016-12-10"
MAINTAINER Carolina Santana "c.santanamartel@gmail.com"
RUN apt-get update && apt-get -y install nodejs && \
    apt-get autoclean && apt-get clean && \
    rm -rf /var/lib/apt/lists/* && \
    rm -f /tmp/* /var/tmp/*
CMD ["/bin/bash"]
\end{code}
\end{codelisting}

%= lang:bash
\begin{code}
$ bundle install
\end{code}

A partir de ahora se ejecuta el script que realizará el despliegue. En primer lugar, se crea la imagen de la aplicación web. En segundo lugar se creará el contenedor \kode{some-postgres} con el servidor \kode{PostgreSQL}, a partir de la imagen \kode{postgres} indicada con \textit{\--d}. También se especifica que redirija el tráfico, con \textit{\--p}, del sistema anfitrión en el puerto 5432 al mismo puerto en el contenedor. Para poder acceder a la base de datos es necesario pasarle las credenciales en el momento de su creación, con \textit{\--e}. Una vez se tiene en ejecución se crea el contenedor de la aplicación web, llamado \kode{some-app}, a partir de la imagen de ésta, \textit{\--d}, enlazando la base de datos de la misma con el contenedor \kode{some-postgres} a través del indicador \textit{\--link}. El indicador \textit{-ti} ofrece un terminal interactivo dentro del contedor y \textit{-w} indica que se va a compartir el directorio inicial del proyecto local en el contenedor. 

Luego se crea un volumen para que los contenedores \kode{some-app} y \kode{some-nginx} compartan el directorio \kode{/usr/src/app/public}.

El contenedor de la aplicación web \kode{some-app}, creado a partir de la imagen \kode{sample\_app\_image\_0\_1} con \textit{\--d}, se enlaza con el contenedor \kode{some-postgres} indicándole que se resolverá como la base de datos a través del indicador \textit{\--link}. Además, se le pasan las variables de entorno necesarias para acceder a la base de datos, con \textit{\--e}. Este contenedor compartirá el directorio de la aplicación en \kode{/usr/src/app}, indicado con \textit{\--w}, así como se le pasa el comando tail para que el contenedor se mantenga en ejecución. Además se monta el volumen \kode{volume-public} con el indicador \textit{\--v}.

Así, se accede al contenedor \kode{some-app} para poder construir, migrar y poblar la base de datos que se ejecuta en el contenedor \kode{some-app}, dentro del contenedor \kode{some-postgres}. 

Para crear el contenedor \kode{some-nginx} que proporciona el servidor \kode{Nginx} se indica la imagen con \textit{\--d}, que el tráfico del puerto 80 en el sistema anfitrión se redirija al 80 del contenedor, \textit{\--p}, y se enlaza la aplicación web con el contenedor en el que se ejecuta, a través del indicador \textit{\--link}. Además, se crea localmente el fichero de configuración \kode{/etc/nginx/conf.d/nginx.conf} en el que se especifica que este servidor escuchará en el puerto 80, compartiendo el directorio \kode{/usr/src/app/public}, también indicado en la creación con \textit{\--w}, así como que ha de resolver la aplicación en el puerto 9292, que es el que usa \kode{puma} para la ejecución. Además se monta el volumen \kode{volume-public} con el indicador \textit{\--v}.

El contenido de este fichero es: 
\begin{codelisting}
\label{code:nginxconf}
\codecaption{Contenido de \kode{nginx.conf}}
%= lang:java
\begin{code}
server {
  listen 80;

  root /usr/src/app/public;
  location / {
    proxy_set_header X-Forwarded-For $proxy_add_x_forwarded_for;
    proxy_set_header Host $http_host;
    proxy_redirect off;
    try_files $uri /page_cache/$uri /page_cache/$uri.html @app;
  }
  
  location @app{
    proxy_pass http://app:9292;
    break;
  }
}
\end{code}
\end{codelisting}

El script utilizado, con permiso \kode{chmod +x} y ejecutado como \kode{./monomachine-docker.sh}, es el siguiente:
\begin{codelisting}
\label{code:nginxconf}
\codecaption{Contenido de \kode{monomachine-docker.sh}}
%= lang:bash
\begin{code}
#!/bin/bash

if [ "$POSTGRES_USER" = "" ] || [ "$POSTGRES_PASSWORD" = "" ]; then
	echo "Environment variables for POSTGRES not found"
        exit
fi

docker build -t sample_app_image_0_1 .

docker run --name some-postgres -e POSTGRES_USER=$POSTGRES_USER \
           -e POSTGRES_PASSWORD=$POSTGRES_PASSWORD -d postgres

docker volume create --name volume-public

docker run -d -it --name some-app -e POSTGRES_USER=$POSTGRES_USER \
           -e POSTGRES_PASSWORD=$POSTGRES_PASSWORD -w /usr/src/app \
           -v volume-public:/usr/src/app/public \
           --link some-postgres:db sample_app_image_0_1 tail -f /dev/null

docker exec -d some-app rake db:create
docker exec -d some-app rake db:migrate
docker exec -d some-app rake db:seed
docker exec -d some-app rake db:populate

docker run --name some-nginx \
           -v "${PWD}/nginx.conf":/etc/nginx/conf.d/default.conf \
           -p 80:80 --link some-app:app -v volume-public:/usr/src/app/public -d nginx

\end{code}
\end{codelisting}

Finalmente, se ejecuta el servidor \kode{puma} dentro del contenedor \kode{some-app} y se comprueba desde el sistema anfitrión que la página principal de la aplicación \kode{sample\_app\_rails\_4} está disponible en el puerto 80 mediante el comando \kode{curl http://localhost:80}, produciendo el siguiente volcado:

\begin{figure}[H]
\image{images/figures/curldocker.png}
\caption{Volcado de ejecución \kode{curl http://localhost:80}.\label{fig:figure_placement_example}}
\end{figure}

