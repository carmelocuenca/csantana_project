\chapter{Análisis de la aplicación Ruby on Rails}
\label{cha:regulation}

En esta sección se expone el análisis de la aplicación web Ruby on Rails, de 3 capas, utilizada para el desarrollo del proyecto.

La aplicación \kode{sample\_app\_rails\_4}, escrita en lenguaje Ruby\footnotettt{Ruby}{https://www.ruby-lang.org/es}, es parte de un tutorial sobre el uso del \textit{framework} de desarrollo de aplicaciones web Ruby on Rails\footnotettt{Ruby on Rails}{http://www.rubyonrails.org.es}. Se trata de una aplicación de muestra, desarrollada mediante una combinación de simulaciones, pruebas de desarrollo \textit{TDD} y pruebas de integración. Tiene una arquitectura de 3 capas: cliente, aplicación y base de datos. Está creada a partir de páginas estáticas con contenido dinámico, tiene un diseño de sitio web, un modelo de datos de usuario y un sistema completo de registro y autenticación, incluida la activación de cuentas y restablecimientos de contraseñas. Además tiene funciones de \textit{microblogging} y redes sociales. Así, tendrá usuarios que crearán \textit{microposts} dentro de un marco de autenticación e inicio de sesión completo.

\begin{figure}[H]
\image{images/figures/sampleapp.png}
\caption{Integración entre características y componentes de Ruby on Rails.\label{fig:figure_placement_example}}
\end{figure}

La arquitectura de Ruby on Rails que implementa la aplicación web tiene las siguientes características:
\begin{itemize}
\item Arquitectura Modelo-Vista-Controlador (MVC): Mejora la capacidad de mantenimiento, desacoplamiento y pruebas de la aplicación.
\subitem-- Modelo: Lógica de negocio de la aplicación y reglas para manipular los datos. Representa la información en la base de datos y realizan las validaciones apropiadas.
\subitem-- Vista: Representa la interfaz de usuario.
\subitem-- Controlador: Responde a eventos e invoca perticiones al modelo cuando se hace una solicitud de información.
\item Arquitectura \textit{Representational State Transfer} (REST) para servicios web.
\item Soporta las principales bases de datos como MySQL\footnotettt{MySQL}{https://www.mysql.com}, Oracle\footnotettt{Oracle}{https://www.oracle.com/es/database/index.html} y PostgreSQL\footnotettt{PostgreSQL}{http://www.postgresql.org.es}, entre otras.
\item Lenguaje \textit{scripting} del lado del servidor de código abierto.
\item Convención sobre configuración.
\item Generadores de \textit{scripts} para automatizar tareas.
\item Uso de \textit{YAML}, formato de serialización de datos inspirado en lenguajes como \textit{XML} y \textit{C}.
\end{itemize}

Las características anteriormente descritas se distribuyen en los siguientes componentes de Rails:
\begin{itemize}
\item \textit{Action Mailer}: Responsable de proporcionar servicios de correo electrónico. 
\item \textit{Action Pack}: Capta las solicitudes de usuario realizadas por el navegador y asignan estas solicitudes a acciones definidas en la capa de controladores.
\subitem-- \textit{Action Controller}: Enruta solicitudes a su controlador correspondiente. 
\subitem-- \textit{Action Dispatcher}: Controla el enrutamiento de la solicitud del navegador web, la analiza y procesa.
\subitem-- \textit{Action View}: Realiza la presentación de la página web solicitada.
\item \textit{Active Model}: Define la interfaz entre el \textit{Action Pack} y los módulos \textit{Active Record}.
\item \textit{Active Record}: Proporciona capacidad de crear relaciones o asociaciones entre modelos y construye la capa Modelo que conecta las tablas de la base de datos con su representación en las clases Ruby.
\item \textit{Active Resource}: Administra la conexión entre servicios web \textit{RESTful} y objetos de negocio.
\item \textit{Active Support}: Colección de clases de utilidad y extensiones de bibliotecas estándar de Ruby útiles para el desarrollo en Ruby on Rails.
\item \textit{Railties}: Código básico de Rails que construye nuevas aplicaciones. \end{itemize}	 

\begin{figure}[H]
\image{images/figures/rubyonrails.png}
\caption{Integración entre características y componentes de Ruby on Rails.\label{fig:figure_placement_example}}
\end{figure}